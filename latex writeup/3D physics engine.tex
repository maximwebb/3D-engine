\documentclass{article}

\usepackage{amsmath}
\usepackage{amssymb}
\usepackage{graphicx}
\usepackage{hyperref}
\usepackage{xcolor}
\usepackage{listings}
\graphicspath{ {images/} }

\lstset{frame=tb,
	language=C++,
  	aboveskip=3mm,
  	belowskip=3mm,
  	showstringspaces=false,
  	columns=flexible,
  	basicstyle={\small\ttfamily},
  	numbers=none,
  	numberstyle=\tiny\color{gray},
  	keywordstyle=\color{blue},
  	commentstyle=\color{brown},
  	stringstyle=\color{mauve},
  	breaklines=true,
  	breakatwhitespace=true,
  	tabsize=3
}


\title{
	\Huge
	{Building a 3D Physics Engine to Model Projectile Motion}\\
}
\author{Maxim Webb}

\begin{document}

\maketitle

\newpage

\section{Abstract}
In this report, I will describe the process of building a 3D physics engine, with the end goal being to create an accurate model for projectile motion.
\newline
\newline
However, this is not the sole goal; the aim throughout the project is to build a physics engine from first principles, without relying on pre-existing 3D graphical APIs such as OpenGL, or Direct3D. Therefore, the report will also describe the mathematics behind the rendering process, as well as the necessary methods for representing 3D transformations.
\newline

\section{Technology}
The program will be written in C++, for its speed, and for its object orientation, which will prove useful for the structure of the project.
\newline
\newline
Windows GDI will be used for basic graphical output, which is limited to drawing lines and polygons. 
\newline
\newline
Git will be used for this project, providing versatile and powerful VCS. The project's code is available 
\href{https://github.com/maximwebb/3D-engine}{\color{blue} on Github}\color{black}.
\newline
\newline

\newpage

\section{Basic Structure}

First off, we create a Point class, with the parameter absPos describing the absolute position in the universe. Also included are the two methods "update" and "draw" -  update will project the point's absolute position to a position on the viewport,
\newline
\begin{lstlisting}
class Point {
public:
	/* x, y, z coordinates */
	vector<float> absPos;
	
	void update() {
	
	}
	
	void draw() {
	
	}
}
\end{lstlisting}

\end{document}